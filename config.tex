%------------ Hinweise --------------------------------------------
% Um das fertige Dokument zu erstellen, wie folgt vorgehen:
% 1. pdflatex Diplomarbeit.tex
% 2. makeindex Diplomarbeit.idx -s Diplomarbeits.ist -t Diplomarbeit.nlg -o Diplomarbeit.nls 
% (Alternativ sollte auch makeindex Diplomarbeit funktionieren)
% 3. makeglossaries Diplomarbeit
% 4. biber Diplomarbeit
% 5. pdflatex Diplomarbeit.tex
%
% Das ganze lässt sich in diversen Editoren (z.B. TeXnicCenter) in fertigen Profilen 
% speicheren bzw. durch leichte Abwandlung schon vorhandener Profile erreichen. Beim
% allerersten Aufruf ist der ganze Ablauf zweimal nötig, um alle Referenzen usw.
% aufzulösen!
%
% Die hier geladenene Pakete stellen viele grundlegende Funktionen zur Verfügung. 
% Alles weitere bitte selbst hinzufügen.


%------------ Header --------------------------------------------
\documentclass[%
  ngerman,%
  BCOR=0mm,
  cdgeometry=yes,
  numbers=noenddot,
  paper=a4,
  listof=totoc, bibliography=totoc,
]{tudscrreprt}
\iftutex
  \usepackage{fontspec}
\else
  \usepackage[T1]{fontenc}
  \usepackage[ngerman=ngerman-x-latest]{hyphsubst}
\fi
\setcounter{secnumdepth}{4} % Ueberschriften bis 4. Ebene nummeriert
\setcounter{tocdepth}{4} 		% Ueberschriften bis 4. Ebene in Inhaltsverzeichnis


%------------ Deutsche Anpassungen und Schrift ------------------------------------
\usepackage[ngerman]{babel}
\usepackage[style=numeric, sorting=none, backend=biber]{biblatex}
%\usepackage[style=ieee, sorting=none, backend=biber]{biblatex}
\usepackage[T1]{fontenc}			 % Font-Encoding, welches Umlaute unterstützt
\usepackage[utf8]{inputenc} % Tex-Dateien werden ANSI-Kodiert erwartet. Umlaute können direkt im Tex-Dokument verwendet werden
\usepackage{textcomp}
\usepackage{gensymb}

\usepackage{csquotes}		% Für biblatex
\usepackage{lmodern}     			 % Umfangreiche und hübsche Schriftart

\usepackage{caption}    			 % Captions außerhalb von floats erstellen
\usepackage{upgreek}		 			 % Griechische Einheiten nicht kursiv darstellen

% \usepackage[default]{opensans}
\usepackage[ngerman]{babel}							% Deutsche Bezeichner
\addto\captionsngerman{%								% Macht aus der Beschriftung Abbildung --> Abb.
	\renewcommand{\figurename}{Abb.}%			%
	\renewcommand{\tablename}{Tab.}%			%
}
\babelhyphenation[ngerman]{pa-ral-lel}
  

%------------ Grafiken und Abbildungen --------------------------------------------
\usepackage[pdftex]{graphicx} % Grafiken in pdfLaTeX
\usepackage{float}						% Das Bild mit [ht] wenn möglich an der Stelle wie im Tex-Dokument ausgeben.
\pdfinclusioncopyfonts=1 			% Richtige Darstellung der in Matlab erstellten Axenbeschriftungen

\usepackage{subcaption}				% Mehrere Bilder in einer Figure Umgebung mit separaten Bildunterschriften

\renewcommand{\thefigure}{\arabic{chapter}.\arabic{figure}}		% Abbildungsnummerierung im Format Kapitel.Nummer


%------------ Formeldarstellung --------------------------------------------
\usepackage{amsmath}     % Mathematische Befehle
\usepackage{amssymb}     % Erweiterte math. Symbole lädt implizit amsfonts
\usepackage{amstext}		 % \text{} im math-modus
\usepackage{siunitx}		 % Korrektes Verwendung von Einheiten mit \SI{Zahl}{\einheit}


%------------ Zeilenabstand und Absätze --------------------------------------------
\usepackage{setspace}		% Zeilenabstand einstellen
%\singlespacing					% 1-zeilig (Standard)
\onehalfspacing					% 1,5-zeilig
%\doublespacing					% 2-zeilig

\setlength{\parindent}{1em}	% Einrücktiefe von neuen Absätzen


%------------ Seitenlayout --------------------------------------------
\usepackage{geometry}                % Ändern der Seitengeometrie. 

%------------ Header und Footer --------------------------------------------
% Fuß- und Kopfseparierungslinie im scrlayer style. Im plain.scrlayer style nur eine Fußlinie
\usepackage[headsepline=0.5pt,footsepline=0.5pt,plainfootsepline]{scrlayer-scrpage}

\KOMAoptions{onpsinit=\setstretch{1.2}} % Verhindert die Beeinflussung von header und footer durch spacing Umgebungen
% \pagestyle{scrheadings}

\ohead{\leftmark}		% Kapitel außen in der Kopfzeile (bei Einsitigem Druck heißt das rechts)
\ifoot{Masterarbeit Antonia Gerdes}

% Seitennummer außen in Fußzeile (wird für Kaptielüberschriften und dergleichen automatisch verwendet)
% '*' heißt, das die Einstellung auch im plain style gilt
\ofoot*{\thepage}
\cfoot*{}			% Im scrlayer Style voreingestellte Seitennummer in der Mitte unten entfernen.
\chead*{} 		% Siehe \cfoot, allerdings für den Header
\setkomafont{pageheadfoot}{\normalsize} 	% Normale Schrift, nicht kursiv wie voreingestellt


%------------ Links im PDF-Dokument -------------------------------------------
\usepackage[hypertexnames=false]{hyperref}%Verlinkungen des TOC im Dokument machen (3xkompilieren)
\hypersetup{
  hypertexnames=false,   % korrekte Numerierung wg. arabisch und römisch
  colorlinks=true,       % Links werden farbig und nicht umrandet
  linkcolor=black,       % Linkfarbe
  citecolor=black,       % Farbe für Zitate
  urlcolor=black,        % Farbe für Webadressen
  bookmarksopen=true,    % im Reader wird Gliederung angezeigt
 %pdfpagelabels=true,    % Nummerierung für PDF-Datei
            } 
						
						
%------------ Abkürzungsverzeichnis --------------------------------------------
\usepackage[acronym,toc,nomain,nonumberlist,nopostdot, nogroupskip]{glossaries}
\renewcommand{\glsnamefont}[1]{\textbf{#1}}
\makeglossaries

%------------ Selbst hinzugefügte Packages -----------------------------------
\usepackage{pdflscape}
\usepackage{booktabs} % Für bessere Tabellen
\usepackage{multirow} % Für Mehrzeilen
\usepackage{array}    % Für Spaltenformatierung
\usepackage{makecell} % Für Umbrüche innerhalb von Zellen in Tabellen
\usepackage{adjustbox}% Um Tabellen zu resizen

%------------ Weitere Packages --------------------------------------------
\usepackage{url}					% Richtige Onlinezitierung

\usepackage{pdfpages}			% Um pdf Dateien dem Latex-dokument EINFACH hinzuzufügen 

\usepackage{color}				% Farbige Markierungen setzen\colorbox{yellow}{hervorgehoben}
\definecolor{dunkelgrau}{rgb}{0.8,0.8,0.8} % Define user colors using the RGB model
\definecolor{hellgrau}{rgb}{0.90,0.90,0.90}
\usepackage{colortbl}			% Um die Farben in Tabellen zu verwenden


\usepackage{remreset} 		% Paket um den Counter für die Fussnoten zu setzen da diese Ansonsten mit jedem neuen Kapitel 
													% wieder von 1 beginnen

\makeatletter           	% Ändern der Vorlage erlauben
\setlength{\@fptop}{0pt}	% floats oben plazieren
\@addtoreset{figure}{chapter}						%Rücksetzen der Abbildungsnummerierung mit jedem neuen Kapitel
\@removefromreset{footnote}{chapter}		%Fussnoten vom Rücksetzen bei Beginn eines neuen Kapitel ausschließen
\makeatother

\Ifpdfoutput{																																			% PDF-Eigenschaften festlegen
    \hypersetup {																																	%
        pdftitle={Titel der Arbeit},	% Titel
        pdfsubject={Masterarbeit},																								% Thema
        pdfauthor={Antonia Gerdes},																								%	Autor
        pdfkeywords={Masterarbeit, TU DD, Level-Set, Segmentierung}								%	Schlüsselwörter
    }
}{}

\pdfminorversion=6
\DeclareMathOperator{\e}{e}

\usepackage{listings}		% Quellcode Listings mit Syntax-Highlighting \begin{lstlisting}...\end{lstlisting}

\usepackage{scrhack}		% Verbessert Kompatibilität zwischen KOMA-Script und anderen Packages (float, hyperrref etc.)

\makeindex							% Let Latex output a .idx file for makeindex