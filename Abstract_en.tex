\chapter*{Abstract}

Atrial fibrillation is one of the most common cardiac arrhythmias and leads to a higher risk of stroke, which can cause permanent damage and premature death. The early detection of atrial fibrillation is desirable, as untreated atrial fibrillation can become chronic. The gold standard for atrial fibrillation diagnosis is the 12-lead ECG, which is interpreted by medical personnel. Because paroxysmal atrial fibrillation only occurs sporadically, a longterm ECG, which can be recorded via mobile ECG patches, is required. 

The manual interpretation of longterm ECGs is resource intensiv, which is why an automated interpretation using deep learning algorithms is necessary. Mobile ECG patches have a reduced lead count and the morphology of their signals is different compared to those of 12-lead ECGs. Due to these differences, available classifiers have a limited usefulness on records of mobile ECG patches. To solve this problem, in this work a deep learning algorithm is developed, which is trained on 12-lead ECGs and is robust against a change in signal morphology.

The developed weighted ensemble of 5 domain adversarial neural networks achieves an F1-score of up to 0,962, a recall of up to 0,955 and a specificity of up to 0,973 when used on as test dataset of 12-lead ECGs. On ECGs taken from the TIMELY project, which are recorded via mobile ECG patches, it achieves an F1-score of up to 0,986, a recall of up to 1,00 and a specificity of up to 0,979.